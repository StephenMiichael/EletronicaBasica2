
\begin{Info}
    % Universidade
    {UNIVERSIDADE DO VALE DO ITAJAÍ}
    % Escola
    {ESCOLA DO MAR, CIÊNCIA E TECNOLOGIA}
    % Curso
    {CURSO DE ENGENHARIA DE COMPUTAÇÃO}
    % Titulo
    {APLICATIVO DE GUIA DIGITAL PARA O MUSEU OCEANOGRÁFICO UNIVALI}
    % Autor
    {Stephen Michael Apolinário}
    % Cidade e Data
    {Itajaí (SC), agosto de 2022}
    % Nome da Área de concentração
    {Tecnologia da informação e comunicação}
    % Orientador(a)
    {Lucas Debatin, M.Sc.}
    % Coorientador(a) <Nome do Coorientador(a)>, <Titulação> %%%%%%%% Se não tiver coorientador deixe vazio
    {}
    \end{Info}
    
    % \begin{Dedicatoria}
    % Este elemento é opcional, mas caso queira fazer uma dedicatória, preencha aqui.
    % \end{Dedicatoria}
    
    % \begin{Agradecimentos}
    % Os agradecimentos são opcionais, mas este é o local adequado para isso.
    % \end{Agradecimentos}
    
    % \begin{Epigrafe}
    % A epígrafe é opcional, mas podes trazer citações, poemas, frases de impacto e citar o(s) autor(es).
    % \end{Epigrafe}
    
    \begin{Resumo}
    
    APOLINÁRIO, Stephen Michael. APLICATIVO DE GUIA DIGITAL PARA O MUSEU
    OCEANOGRÁFICO UNIVALI. Itajaí, 2022. \pageref{LastPage} f. Trabalho Técnico científico de Conclusão de Curso (Graduação em Engenharia da Computação) -- Escola do Mar, Ciência e Tecnologia, Universidade do Vale do Itajaí, Itajaí, 2022.
       
    O mercado de aplicativos está em constante expansão, recebendo impulso significativo devido ao Covid-19. Grande parcela dos aplicativos refere-se a categoria de educação e atualmente, a maior parte de aplicativos móveis e sites de museus não contam com acessibilidades a PCDs (Pessoa com Deficiência) visuais. A motivação deste trabalho foi o quadro geral na falta de inclusão de PCDs em museus. Diante este cenário, identifica-se a necessidade de inclusão de PCDs visuais bem como criar uma maior interatividade com o público geral em museus, então surge a oportunidade de desenvolvimento deste guia digital. Este trabalho tem como foco a contribuição na área de Tecnologia da informação e comunicação, com foco na acessibilidade e interatividade. Para estes objetivos, foi desenvolvido um aplicativo para visitantes, e sistema web para administradores. Através dos testes do aplicativo acerca da acessibilidade para PCDs visuais e de usabilidade para o visitante geral como também da web para os administradores, será permitido a análise de melhorias a serem implementadas no TCC III.

    Palavras-chave: Guia Digital. Acessibilidade. Aplicativo móvel. Museu. 
    
    \end{Resumo}
    
    \begin{Abstract}
    
    Este objetivo só será escrito quando o em português estiver finalizado.
    
    Keywords: Digital Guide. Accessibility. Mobile Application.

    \end{Abstract}
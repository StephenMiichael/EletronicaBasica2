\documentclass[
	% -- opções da classe memoir --
	12pt,				% tamanho da fonte
	openright,			% capítulos começam em pág ímpar (insere página vazia caso preciso)
	oneside,			% para impressão em um lado. Oposto a twoside, verso e anverso
	a4paper,			% tamanho do papel. 
	% -- opções da classe abntex2 --
	chapter=TITLE,		% títulos de capítulos convertidos em letras maiúsculas
	section=TITLE,		% títulos de seções convertidos em letras maiúsculas
	%subsection=TITLE,	% títulos de subseções convertidos em letras maiúsculas
	%subsubsection=TITLE,% títulos de subsubseções convertidos em letras maiúsculas
	% -- opções do pacote babel --
	english,			% idioma adicional para hifenização
	%french,				% idioma adicional para hifenização
	spanish,			% idioma adicional para hifenização
	brazil,				% o último idioma é o principal do documento
	]{styles/abntex2}

\usepackage{titlesec}
\usepackage[alf,abnt-etal-list=0,abnt-etal-cite=3,abnt-repeated-author-omit=no,abnt-doi=expand,abnt-emphasize=bf]{styles/abntex2cite}	% Citações padrão ABNT
\usepackage[utf8]{inputenc}
\usepackage[T1]{fontenc}
\makeatletter
\usepackage[singlelinecheck=false,justification=raggedright]{caption}
\usepackage{lastpage}
\usepackage{tabularx}
\usepackage{times}
\usepackage{graphicx}
\usepackage{styles/ttc_univali}
\usepackage{lipsum}
\usepackage{url}
\usepackage[table,xcdraw]{xcolor}
\usepackage{float}
\usepackage{accents}
\usepackage{array}
\usepackage{hyphenat}
\usepackage{multirow}
\usepackage{cancel}
\newcolumntype{C}[1]{>{\centering\arraybackslash}m{#1}}
\DeclareCaptionType{Resolucao}

%\newcolumntype{C}[1]{>{\centering\let\newline\\\arraybackslash\hspace{0pt}}m{#1}
% \usepackage[
%     % showframe=true
%     ]
% {geometry}

\title{TTC Engenharia da Computação}
\author{Modelo from Stephen Michael Apolinário}
\date{August 2022}

\hyphenation{op-tical net-works customiza-cao implementa-cao quan-do microcon-troladores} 

\begin{document}
\titleformat*{\subsubsection}{\normalfont}
\renewcommand{\arraystretch}{1.10}
\pretextual

\begin{Info}
    % Universidade
    {UNIVERSIDADE DO VALE DO ITAJAÍ}
    % Escola
    {ESCOLA DO MAR, CIÊNCIA E TECNOLOGIA}
    % Curso
    {CURSO DE ENGENHARIA DE COMPUTAÇÃO}
    % Titulo
    {APLICATIVO DE GUIA DIGITAL PARA O MUSEU OCEANOGRÁFICO UNIVALI}
    % Autor
    {Stephen Michael Apolinário}
    % Cidade e Data
    {Itajaí (SC), agosto de 2022}
    % Nome da Área de concentração
    {Tecnologia da informação e comunicação}
    % Orientador(a)
    {Lucas Debatin, M.Sc.}
    % Coorientador(a) <Nome do Coorientador(a)>, <Titulação> %%%%%%%% Se não tiver coorientador deixe vazio
    {}
    \end{Info}
    
    % \begin{Dedicatoria}
    % Este elemento é opcional, mas caso queira fazer uma dedicatória, preencha aqui.
    % \end{Dedicatoria}
    
    % \begin{Agradecimentos}
    % Os agradecimentos são opcionais, mas este é o local adequado para isso.
    % \end{Agradecimentos}
    
    % \begin{Epigrafe}
    % A epígrafe é opcional, mas podes trazer citações, poemas, frases de impacto e citar o(s) autor(es).
    % \end{Epigrafe}
    
    \begin{Resumo}
    
    APOLINÁRIO, Stephen Michael. APLICATIVO DE GUIA DIGITAL PARA O MUSEU
    OCEANOGRÁFICO UNIVALI. Itajaí, 2022. \pageref{LastPage} f. Trabalho Técnico científico de Conclusão de Curso (Graduação em Engenharia da Computação) -- Escola do Mar, Ciência e Tecnologia, Universidade do Vale do Itajaí, Itajaí, 2022.
       
    O mercado de aplicativos está em constante expansão, recebendo impulso significativo devido ao Covid-19. Grande parcela dos aplicativos refere-se a categoria de educação e atualmente, a maior parte de aplicativos móveis e sites de museus não contam com acessibilidades a PCDs (Pessoa com Deficiência) visuais. A motivação deste trabalho foi o quadro geral na falta de inclusão de PCDs em museus. Diante este cenário, identifica-se a necessidade de inclusão de PCDs visuais bem como criar uma maior interatividade com o público geral em museus, então surge a oportunidade de desenvolvimento deste guia digital. Este trabalho tem como foco a contribuição na área de Tecnologia da informação e comunicação, com foco na acessibilidade e interatividade. Para estes objetivos, foi desenvolvido um aplicativo para visitantes, e sistema web para administradores. Através dos testes do aplicativo acerca da acessibilidade para PCDs visuais e de usabilidade para o visitante geral como também da web para os administradores, será permitido a análise de melhorias a serem implementadas no TCC III.

    Palavras-chave: Guia Digital. Acessibilidade. Aplicativo móvel. Museu. 
    
    \end{Resumo}
    
    \begin{Abstract}
    
    Este objetivo só será escrito quando o em português estiver finalizado.
    
    Keywords: Digital Guide. Accessibility. Mobile Application.

    \end{Abstract} \clearpage
% ---
% inserir lista de ilustrações
% ---
\pdfbookmark[0]{\listfigurename}{lof}
\listoffigures*
\cleardoublepage
% ---

% ---
% inserir lista de tabelas
% ---
% \pdfbookmark[0]{\listtablename}{lot}
% \listoftables*
% \cleardoublepage
% ---

% ---
% inserir lista de quadros
% ---
\pdfbookmark[0]{\listofquadrosname}{loq}
\listofquadros*
\cleardoublepage
% ---

% ---
% inserir lista de abreviaturas e siglas
% ---
% \begin{siglas}
%     \item[PCD] Pessoa com deficiência
%     \item[PWD] Person with disability
%     \item[MOVI] Museu Oceanográfico Univali
%     \item[MVP]  Minimum Viable Product
%     \item[TIC] Trabalho de iniciação cientifica 

% \end{siglas}
% ---

% ---
% inserir lista de símbolos
% ---
%\begin{simbolos}
%  \item[$ \Gamma $] Letra grega Gama
%  \item[$ \Lambda $] Lambda
%  \item[$ \zeta $] Letra grega minúscula zeta
% \item[$ \in $] Pertence
%\end{simbolos}
% ---

% ---
% inserir o sumario
% ---
\pdfbookmark[0]{\contentsname}{toc}
\tableofcontents*
\cleardoublepage
% --- \clearpage

\textual % Indica inicio dos elementos textuais
\pagestyle{simple} % remove o cabeçalho
\chapter{OBJETIVOS}

Os objetivos deste relatório possuem obter o conhecimento dos seguintes tópicos:

\begin{enumerate}
    \item Análise de circuitos com diodos
    \item Retificadores de onda completa
    \item Filtros capacitivos
\end{enumerate} \clearpage
\chapter{INTRODUÇÃO}

Os diodos abordados nas aulas referenciadas neste relatório, são componentes eletrônicos que possuem a função de permitir a passagem de corrente elétrica em apenas uma direção. Sendo assim, os diodos são componentes semicondutores que possuem uma característica de polarização, ou seja, a passagem de corrente elétrica só ocorre quando o diodo é polarizado. Estes componentes são muito importantes para a eletrônica, pois através deles é possível criar circuitos mais complexos com a utilização de outros componentes eletrônicos. \clearpage
\chapter{DESENVOLVIMENTO}

\section{Circuito retificador em ponte}

\begin{figure}[H]
    \centering
    \fbox{
        \parbox{0.975\textwidth}{
            \centering
            \includegraphics[]{images/circuito_slide_01.png}
        }}
    \caption{Circuito 01}
    \vspace{-0.3cm}
    \label{fig:Circuito01}
\end{figure}

Considerando o circuito da figura \ref{fig:Circuito01}, temos que:

\begin{Resolucao}[H]
    \fbox{
        \parbox{0.975\textwidth}{
            \vspace{0.40cm}
            \centering
            \[Vp = \textcolor{red}{311V}\]
            \[n = \frac{1}{60} = \textcolor{red}{0.0166\dots}\]
            \[Vpico = Vp * n \]

            \[\textnormal{Tensão eficaz no primário de T1: } \frac{311}{\sqrt[2]{2}} = \textcolor{red}{220V}\]
            \[\textnormal{Tensão eficaz no secundário de T1: } \frac{Eficaz1}{Fator} \rightarrow \frac{220}{60} = \textcolor{red}{3.67V} \]
            \[\textnormal{Tensão de pico de entrada no secundário: } Vpico = \frac{311}{60} \simeq \textcolor{red}{5.18V}\]
            \[\textnormal{Tensão média na saída: }  \frac{Vpico - (0.7 * 2)}{\pi} * 2 \simeq \textcolor{red}{2.41V}\]
            \[\textnormal{Tensão de pico na saída: } Vpico - (0.7 * 2) = \textcolor{red}{3,79V}\]
            \[\textnormal{Tensão reversa sobre o diodo: } \textcolor{red}{-4.49V}\]
            \[\textnormal{Corrente média na saída: }  \frac{TensaoMédia}{Ressistencia} \rightarrow \frac{2.41}{5} \simeq \textcolor{red}{0.482A}\]
            \[\textnormal{Corrente de pico no diodo: } \frac{TensãoDePico}{Carga} = \frac{3.79}{5} = \textcolor{red}{0.758A} \]
            \textcolor{red}{\textbf{OBSERVAÇÃO:}} \textcolor{red}{Pode-se observar que a frequência na saída é o dobro, e o período é a metade da entrada.}

            
            % \[\textnormal{Tensão de pico de entrada no primeiro: } \textcolor{red}{311V}\]
            % \[N = \textcolor{red}{\frac{1}{60}}\]
            % \[\textnormal{Tensão de pico na entrada do secundário: } Vf * n * \frac{1}{60} = \textcolor{red}{5.18V}\]
            % \[\textnormal{Tensão de pico na saída: } 5.18 - 1.4 = \textcolor{red}{3.78V}\]
        }
    }
    \captionof*{Resolucao}{Resolução: Circuito Exercício 1}
    \label{res:circuito01}
\end{Resolucao}

\begin{figure}[H]
    \centering
    \fbox{
        \parbox{0.975\textwidth}{
            \centering
            \includegraphics[]{images/simulacoes/circuito_01.png}
        }}
    \caption{Simulação: Circuito 01}
    \vspace{-0.3cm}
    \label{fig:SimulacaoCircuito01}
\end{figure}

\begin{figure}[H]
    \centering
    \fbox{
        \parbox{0.975\textwidth}{
            \centering
            \includegraphics[width=0.975\textwidth]{images/simulacoes/osciloscopio_circuito01.png}
        }}
    \caption{Osciloscópio: Circuito 01}
    \vspace{-0.3cm}
    \label{fig:OsciloscopioCircuito01}
\end{figure}

Com a tabela \ref{tab:Comparacao1Circuito} podemos comparar os resultados obtidos por simulação com os resultados obtidos por cálculo, na qual comprovam que os cálculos estavam corretos.

\begin{quadro}[H]
    \centering
    \caption{Comparação entre os resultados obtidos por simulação e os resultados obtidos por cálculo do circuito 01}
    \begin{tabular}{|C{0.19\textwidth}|C{0.24\textwidth}|C{0.24\textwidth}|C{0.24\textwidth}|C{0.24\textwidth}|}
        \hline
        \rowcolor[HTML]{C0C0C0}
        \textbf{Modelo\textbackslash{}Variáveis} & \textbf{Tensão de pico de entrada no 2} & \textbf{Tensão de pico de saída no 2} \\
        \hline
        Calculado & 5.18V & 3.78V \\
        \hline
        Simulado & 4.401V & 3.495V \\
        \hline
    \end{tabular}
    \vspace{-0.6cm}
    \label{tab:Comparacao1Circuito}
\end{quadro}

O objetivo da ponte retificadora é usar uma ponte de diodos para que em cada semiciclo da fonte, 2 diodos começaram a conduzir, gerando cada par uma onda, onde juntos formam uma onda completa. Cada diodo tem uma queda de tensão de 0,7v.

\section{Retificador de meia onda com filtro capacitivo}

\begin{figure}[H]
    \centering
    \fbox{
        \parbox{0.975\textwidth}{
            \centering
            \includegraphics[]{images/circuito_slide_02.png}
        }}
    \caption{Circuito 02}
    \vspace{-0.3cm}
    \label{fig:Circuito02}
\end{figure}

Considerando o circuito da figura \ref{fig:Circuito02}, temos que:

\begin{Resolucao}[H]
    \fbox{
        \parbox{0.975\textwidth}{
            \vspace{0.40cm}
            \centering
            \[\textnormal{Tensão de pico na saída: } Vi - 0.7V \rightarrow 100-0.7 = \textcolor{red}{99.3V}\]
            \[\textnormal{A tensão de ripple: } Vr = \frac{vp}{Frc} \rightarrow \frac{99.3}{60*100K*100u} = \textcolor{red}{1.655V}\]
            \[\textnormal{A tensão média na carga: } Vdc = Vp - \frac{vr}{2} \rightarrow 99.3 - 0.8275 = \textcolor{red}{98.473V}\]
            \[Vrms = \frac{100}{\sqrt[2]{2}} = \textcolor{red}{70.7106}\]
        }
    }
    \captionof*{Resolucao}{Resolução: Circuito Exercício 2}
    \label{res:circuito02}
\end{Resolucao}

\begin{figure}[H]
    \centering
    \fbox{
        \parbox{0.975\textwidth}{
            \centering
            \includegraphics[]{images/simulacoes/circuito_02.png}
        }}
    \caption{Simulação: Circuito 02}
    \vspace{-0.3cm}
    \label{fig:SimulacaoCircuito02}
\end{figure}

\begin{figure}[H]
    \centering
    \fbox{
        \parbox{0.975\textwidth}{
            \centering
            \includegraphics[width=0.975\textwidth]{images/simulacoes/osciloscopio_circuito02.png}
        }}
    \caption{Osciloscópio: Circuito 02}
    \vspace{-0.3cm}
    \label{fig:OsciloscopioCircuito02}
\end{figure}

\begin{figure}[H]
    \centering
    \fbox{
        \parbox{0.975\textwidth}{
            \centering
            \includegraphics[width=0.975\textwidth]{images/simulacoes/osciloscopio_circuito02_ripple.png}
        }}
    \caption{Osciloscópio: Circuito 02 - Ripple}
    \vspace{-0.3cm}
    \label{fig:OsciloscopioCircuito02Ripple}
\end{figure}

A forma de onda da tensão de ripple, é a diferença entre os dois ponteiros do osciloscópio, que é a diferença entre os dois semiciclos da onda de entrada, onde o primeiro ponteiro é a tensão de pico de entrada e o segundo ponteiro é a tensão de pico de saída. A imagem \ref{fig:OsciloscopioCircuito02Ripple} mostra a tensão de ripple.

Com a tabela \ref{tab:Comparacao2Circuito} podemos comparar os resultados obtidos por simulação com os resultados obtidos por cálculo, na qual comprovam que os cálculos estavam corretos.

\begin{quadro}[H]
    \centering
    \caption{Comparação entre os resultados obtidos por simulação e os resultados obtidos por cálculo do circuito 02}
    \begin{tabular}{|C{0.19\textwidth}|C{0.24\textwidth}|C{0.24\textwidth}|C{0.24\textwidth}|C{0.24\textwidth}|}
        \hline
        \rowcolor[HTML]{C0C0C0}
        \textbf{Modelo\textbackslash{}Variáveis} & \textbf{Tensão de pico de entrada} & \textbf{Tensão saída} & \textbf{Tensão de ripple}\\
        \hline
        Calculado & 100V & 99.3V & 1.65V \\
        \hline
        Simulado & 99.482V & 99.075V & 1.328V \\
        \hline
    \end{tabular}
    \vspace{-0.6cm}
    \label{tab:Comparacao2Circuito}
\end{quadro}

Analisando a forma de onda de tensão de saída do filtro capacitivo, o capacitor descarrega durante o período em que a fonte alimenta o circuito com valor menor de tensão do que o capacitor possui, podemos ver a descarga do capacitor, quando a fonte atinge novamente o valor de tensão maior que a carga contida no capacitor, o capacitor começa a se carregar novamente.

\section{Retificador em ponte com filtro capacitivo}

\begin{figure}[H]
    \centering
    \fbox{
        \parbox{0.975\textwidth}{
            \centering
            \includegraphics[]{images/circuito_slide_03.png}
        }}
    \caption{Circuito 03}
    \vspace{-0.3cm}
    \label{fig:Circuito03}
\end{figure}

Considerando o circuito da figura \ref{fig:Circuito03}, temos que:

\begin{Resolucao}[H]
    \fbox{
        \parbox{0.975\textwidth}{
            \vspace{0.40cm}
            \centering
            \[\textnormal{Tensão de pico na saída: } Vi - 1.4V \rightarrow 100 - 1.4 = \textcolor{red}{98.6V}\]
            \[\textnormal{Tensão de ripple: } Vr = \frac{Vp}{Frc} \rightarrow \frac{98.6}{120*10K*100u} = \textcolor{red}{0.8216V}\]
            \[\textnormal{Tensão média na carga: } Vp(saida) - \frac{Vr}{2} \rightarrow 98.6 - \frac{0.8216}{2} = \textcolor{red}{98.189V}\]
            \[Vrms = \frac{100}{\sqrt[2]{2}} \simeq \textcolor{red}{70.7106V}\]
        }
    }
    \captionof*{Resolucao}{Resolução: Circuito Exercício 3}
    \label{res:circuito03}
\end{Resolucao}

\begin{figure}[H]
    \centering
    \fbox{
        \parbox{0.975\textwidth}{
            \centering
            \includegraphics[]{images/simulacoes/circuito_03.png}
        }}
    \caption{Simulação: Circuito 03}
    \vspace{-0.3cm}
    \label{fig:SimulacaoCircuito03}
\end{figure}

\begin{figure}[H]
    \centering
    \fbox{
        \parbox{0.975\textwidth}{
            \centering
            \includegraphics[width=0.975\textwidth]{images/simulacoes/osciloscopio_circuito03.png}
        }}
    \caption{Osciloscópio: Circuito 03}
    \vspace{-0.3cm}
    \label{fig:OsciloscopioCircuito03}
\end{figure}

\begin{figure}[H]
    \centering
    \fbox{
        \parbox{0.975\textwidth}{
            \centering
            \includegraphics[width=0.975\textwidth]{images/simulacoes/osciloscopio_circuito03_ripple.png}
        }}
    \caption{Osciloscópio: Circuito 03 - Ripple}
    \vspace{-0.3cm}
    \label{fig:OsciloscopioCircuito03Ripple}
\end{figure}

Com a tabela \ref{tab:Comparacao3Circuito} podemos comparar os resultados obtidos por simulação com os resultados obtidos por cálculo, na qual comprovam que os cálculos estavam corretos.

\begin{quadro}[H]
    \centering
    \caption{Comparação entre os resultados obtidos por simulação e os resultados obtidos por cálculo do circuito 03}
    \begin{tabular}{|C{0.19\textwidth}|C{0.24\textwidth}|C{0.24\textwidth}|C{0.24\textwidth}|C{0.24\textwidth}|}
        \hline
        \rowcolor[HTML]{C0C0C0}
        \textbf{Modelo\textbackslash{}Variáveis} & \textbf{Tensão de pico de entrada} & \textbf{Tensão saída} & \textbf{Tensão de ripple}\\
        \hline
        Calculado & 100V & 98.6V & 0.8216V \\
        \hline
        Simulado & 99.264V & 98.998V & 0.605V \\
        \hline
    \end{tabular}
    \vspace{-0.6cm}
    \label{tab:Comparacao3Circuito}
\end{quadro}

Avaliando então percebermos que a frequência de saída do circuito é o dobro da frequência do sinal de entrada. \clearpage
\chapter{CONCLUSÃO}

Através dos experimentos é possível analisar as diferenças de cálculo encontradas quando realizado os cálculos teóricos e os valores simulados. Também foi possível verificar o funcionamento do regulador de tensão nos dois modelos utilizados, o simplificado e o Linear.

Na qual, Diodo Zener (também conhecido como diodo regulador de tensão , diodo de tensão constante, diodo de ruptura ou diodo de condução reversa) é um dispositivo ou componente eletrônico semelhante a um diodo semicondutor, especialmente projetado para trabalhar sob o regime de condução inversa, ou seja, acima da tensão de ruptura da junção PN, neste caso há dois fenômenos envolvidos: o efeito Zener e o efeito avalanche. O dispositivo leva o nome em homenagem a Clarence Zener, que descobriu esta propriedade elétrica. \clearpage

\postextual

\bibliography{refs}

\end{document}


\begin{Info}
    % Universidade
    {UNIVERSIDADE DO VALE DO ITAJAÍ}
    % Escola
    {ESCOLA DO MAR, CIÊNCIA E TECNOLOGIA}
    % Curso
    {CURSO DE ENGENHARIA DE COMPUTAÇÃO}
    % Titulo
    {RELATÓRIO 02}
    % Autor
    {Stephen Michael Apolinário}
    % Cidade e Data
    {Itajaí (SC), agosto de 2022}
    % Nome da Área de concentração
    {Análise de circuitos com díodos}
    % Orientador(a)
    {Walter Gontijo}
    % Coorientador(a) <Nome do Coorientador(a)>, <Titulação> %%%%%%%% Se não tiver coorientador deixe vazio
    {}
    \end{Info}
    
    % \begin{Dedicatoria}
    % Este elemento é opcional, mas caso queira fazer uma dedicatória, preencha aqui.
    % \end{Dedicatoria}
    
    % \begin{Agradecimentos}
    % Os agradecimentos são opcionais, mas este é o local adequado para isso.
    % \end{Agradecimentos}
    
    % \begin{Epigrafe}
    % A epígrafe é opcional, mas podes trazer citações, poemas, frases de impacto e citar o(s) autor(es).
    % \end{Epigrafe}
    
    \begin{Resumo}
    
    APOLINÁRIO, Stephen Michael. RELATÓRIO REFERENTE A M1 DE ELETRÔNICA BÁSICA. Itajaí, 2022. \pageref{LastPage} f. Engenharia De Computação, Escola do Mar, Ciência e Tecnologia, Universidade do Vale do Itajaí, Itajaí, 2022.
       
    Neste relatório, será tratado os conteúdos vistos na aula de dia 12 de agosto de 2022, na Univali Itajaí, durante a matéria de Eletrônica Básica, ministrada pelo professor Walter Antonio Gontijo, na qual foi abordado a funcionalidade dos dois modelos de díodo: (i) Díodo ideal, (ii) sua composição (Anodo e Catodo), e (iii) seu comportamento.
    
    Palavras-chave: Díodo ideal, Anodo, Catodo.
    
    \end{Resumo}
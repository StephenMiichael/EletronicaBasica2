\chapter{CONCLUSÃO}

Através deste relatório, o autor conseguiu entender o funcionamento de um diodo de forma básica, onde o mesmo tem o funcionamento de chave fechada ou aberta, mas através disso, foi possível entender, sua composição, além de entender o comportamento seu comportamento e sua aplicação em circuitos. Sendo assim, o diodo é um componente eletrônico que permite a passagem da corrente em somente um sentido.

Pode-se afirmar que o diodo possui diversas aplicações, e uma delas é atuar como um retificador, convertendo tensão alternada em continua. Porém, deve-se atentar que um diodo possui energia dissipada em formato de calor, e por isso, deve-se ter cuidado com a quantidade de corrente que passa por ele, pois pode causar um superaquecimento, e consequentemente, danificar o componente. Além disso, o diodo possui uma aplicação em circuitos de proteção, onde o mesmo pode ser utilizado para proteger circuitos de tensões altas, como por exemplo, em circuitos de proteção de baterias, onde o mesmo pode ser utilizado para proteger o circuito de uma bateria de um curto-circuito, ou até mesmo, de uma descarga excessiva.
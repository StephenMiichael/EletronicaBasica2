\chapter{CONCLUSÃO}

Neste relatório podemos verificar o comportamento do diodo nos circuitos ceifador e grampeador. Podemos observar e analisar a saída de cada circuito, que o ceifador tem como finalidade cortar um sinal de forma que a onda gerada na saída seja a metade do ciclo do sinal de entrada, e o grampeador gera em sua saída um sinal com amplitude definida. ou seja, grampeia o sinal em corrente contínua de forma que a forma de onda não seja afetada.

Podemos concluir com as análises realizadas neste documento que um ceifador corta uma parte do sinal de entrada e passa uma onda em sua saída que seja abaixo ou acima de um valor definido no projeto. As aplicações incluem a limitação de amplitudes excessivas, formação de ondas e o controle da quantidade de potência entregue a uma carga.

Os grampeadores são circuitos com diodos e capacitores que tem por finalidade levar um sinal da entrada para saída, abaixo ou acima de um determinado nível, dependendo ou não se o mesmo for polarizado.


Um ceifador de sinal elimina parte de uma onda e passa somente o sinal que ocorre acima ou abaixo de um determinado nível de tensão ou de corrente. 

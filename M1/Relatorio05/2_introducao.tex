\chapter{INTRODUÇÃO}

Neste relatório, será abordado o funcionamento de retificadores, retificador de meia onda e retificador de onda completa, e qual é o seu papel nos circuitos. O retificador de meia-onda consiste em um circuito para remover metade de um sinal AC (corrente alternada) de entrada, transformando-o em um sinal CC (corrente contínua). É constituído basicamente de um transformador, um diodo e uma carga. Um retificador de onda completa ou um retificador em ponte é equivalente a dois retificadores de meia onda voltados um de costas pro outro, com um retificador controlando o primeiro semiciclo e o outro o semiciclo alternado. Por causa do enrolamento do secundário com derivação central, cada circuito do diodo recebe apenas metade da tensão do secundário.

\chapter{CONCLUSÃO}

Neste relatório, foi abordado o funcionamento de circuitos retificadores, na qual fizemos diversas simulações e cálculos para retificadores de meia onda e de ondas completas, além de um com terminal central. Um circuito retificador, ou simplesmente retificador, corresponde aos circuitos elétricos de tensão elaborados para a conversão de corrente alternada em corrente de passagem.

Aprendemos que um retificador de meia onda é um tipo simples de retificador feito com diodo simples que é conectado em série com carga. Para pequenos níveis de potência, este tipo de circuito retificador é comumente usado. Durante a metade positiva da entrada CA, o diodo torna-se polarizado para frente e as correntes começam a fluir através dele. Durante a metade negativa da entrada AC, o diodo torna-se polarizado inversamente e a corrente para de fluir através dele. Devido ao alto conteúdo de ondulação na saída, este modelo de retificador é raramente usado com carga resistiva.

Também podemos concluir que um retificador de onda central com onda completa utiliza usa dois diodos e um transformador com enrolamento secundário com derivação central. O fluxo de corrente através da carga não muda, mesmo quando a polaridade da tensão muda. Um retificador de onda completa converte toda a forma de onda de entrada em uma polaridade constante (positiva ou negativa) na saída, invertendo as partes negativas (ou positivas) da forma de onda de entrada. As porções negativas ou positivas são então combinadas com o inverso positivo ou negativo para produzir uma forma de onda parcialmente negativa ou positiva.  Existem retificadores de onda completa por dois diodos com transformador de ponto médio e o tipo de retificador de onda completa ponte dupla de Graetz. A corrente obtida na saída dos retificadores não é propriamente contínua e está longe de ser constante e aceitável, o que a tornaria inutilizável para a maioria das aplicações eletrônicas. Para evitar essa desvantagem, a filtragem é realizada para eliminar a ondulação do sinal pulsado retificado.

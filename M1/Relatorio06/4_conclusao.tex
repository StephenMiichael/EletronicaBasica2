\chapter{CONCLUSÃO}

Neste relatório vimos o circuito retificador de onda completa, sua função é manter os níveis de tensão dos dispositivos elétricos e eletrônicos estáveis enquanto a moto estiver ligada e variando a aceleração. Os Filtros são circuitos que permitem filtrar determinadas frequências de um sinal CA permitindo a passagem de algumas frequências e limitando a passagem de outras. 

O circuito retificador de onda completa é o mais empregado nos equipamentos eletrônicos, pois permite obter um melhor aproveitamento da energia disponível na entrada do circuito.

Os filtros servem para regular e estabilizar a tensão de entrada da energia elétrica nos equipamentos ligados a ele, como monitores, aparelhos de som, computadores, impressoras e outros periféricos, evitando assim os picos de tensão. A frequência de transição entre as frequências permitidas e as não permitidas é chamada frequência de corte.

Um filtro capacitivo é um arranjo de circuito elétrico que tem a finalidade de reduzir variações de tensão e corrente de altas frequências. Basicamente os filtros capacitivos usados em fontes servem para eliminar uma tensão alternada pulsativa e transformá-la em uma (tensão contínua) que varia menos. Essa variação é chamada de tensão de ondulação ou ripple. Usando um filtro capacitivo em um circuito retificador, obtém-se uma tensão de ripple resultante do descarregamento lento do capacitor em relação à fonte. O dimensionamento do capacitor utilizado no filtro pode ser feito para gerar uma tensão de ripple controlada para ser posteriormente eliminada através de regulador Zener, regulador linear ou outros tipos de regulagem.
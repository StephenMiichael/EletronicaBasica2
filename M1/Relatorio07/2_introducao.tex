\chapter{INTRODUÇÃO}

O presente relatório será observado o comportamento do diodo Zener usando os modelos de avaliação Simplificado e Linear, e o funcionamento do Regulador de tensão. 
Diodo Zener (também conhecido como diodo regulador de tensão , diodo de tensão constante, diodo de ruptura ou diodo de condução reversa) é um dispositivo ou componente eletrônico semelhante a um diodo semicondutor, especialmente projetado para trabalhar sob o regime de condução inversa, ou seja, acima da tensão de ruptura da junção PN, neste caso há dois fenômenos envolvidos: o efeito Zener e o efeito avalanche. 

O dispositivo leva o nome em homenagem a Clarence Zener, que descobriu esta propriedade elétrica. Para que o diodo Zener opere adequadamente como regulador de tensão é necessário introduzir um resistor que limite a corrente inversa através do diodo a um nível inferior ao valor máximo especificado pelo fabricante. O diodo deve ser conectado em paralelo com a carga, que fica assim submetida à mesma tensão existente entre os terminais do Zener.

Será realizado as comparações do comportamento do Regulador entre os modos teóricos e práticos no qual serão simulados no programa Multisim. 

\chapter{CONCLUSÃO}

Através dos experimentos é possível analisar as diferenças de cálculo encontradas quando realizado os cálculos teóricos e os valores simulados. Também foi possível verificar o funcionamento do regulador de tensão nos dois modelos utilizados, o simplificado e o Linear.

Na qual, Diodo Zener (também conhecido como diodo regulador de tensão , diodo de tensão constante, diodo de ruptura ou diodo de condução reversa) é um dispositivo ou componente eletrônico semelhante a um diodo semicondutor, especialmente projetado para trabalhar sob o regime de condução inversa, ou seja, acima da tensão de ruptura da junção PN, neste caso há dois fenômenos envolvidos: o efeito Zener e o efeito avalanche. O dispositivo leva o nome em homenagem a Clarence Zener, que descobriu esta propriedade elétrica.